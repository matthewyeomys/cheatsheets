%%%%%%%%%%%%%%%%%%%%%%%%%%%%%%%%%%%%%%%%%%%%%%%%%%%%%%%%%%%%%%%%%%%%%%
% Original Source: Dave Richeson (divisbyzero.com), Dickinson College
% Adapted from Chen Yiyang
% Modified By: Matthew Yeo
% 
% A one-size-fits-all LaTeX cheat sheet. Kept to two pages, so it 
% can be printed (double-sided) on one piece of paper
% 
% Feel free to distribute this example, but please keep the referral
% to divisbyzero.com
% 
% Guidance on the use of the Overleaf logos can be found here:
% https://www.overleaf.com/for/partners/logos 
%%%%%%%%%%%%%%%%%%%%%%%%%%%%%%%%%%%%%%%%%%%%%%%%%%%%%%%%%%%%%%%%%%%%%%

\documentclass[10pt,landscape,letterpaper]{article}
\usepackage{amssymb}
\usepackage{amsmath}
\usepackage{amsthm}
\usepackage{physics}  % for vectors
\usepackage{bbm}  % for mathbb-ed digits
%\usepackage{fonts}
\usepackage{multicol,multirow}
\usepackage{spverbatim}
\usepackage{graphicx}
\usepackage{ifthen}
\usepackage[landscape]{geometry}
\usepackage[colorlinks=true,urlcolor=olgreen]{hyperref}
\usepackage{booktabs}
\usepackage{fontspec}
\setmainfont[Ligatures=TeX]{TeX Gyre Pagella}
\setsansfont{Fira Sans}
\setmonofont{Inconsolata}
\usepackage{unicode-math}
\usepackage{listings}
\usepackage{minted}
\setmathfont{TeX Gyre Pagella Math}
\usepackage{microtype}
\usepackage{ulem}  % cuz \underline affected by everymath
\usepackage{empheq}

% new:
\def\MT@is@uni@comp#1\iffontchar#2\else#3\fi\relax{%
  \ifx\\#2\\\else\edef\MT@char{\iffontchar#2\fi}\fi
}
\makeatother

\ifthenelse{\lengthtest { \paperwidth = 11in}}
    { \geometry{margin=0.4in} }
	{\ifthenelse{ \lengthtest{ \paperwidth = 297mm}}
		{\geometry{top=1cm,left=1cm,right=1cm,bottom=1cm} }
		{\geometry{top=1cm,left=1cm,right=1cm,bottom=1cm} }
	}
\pagestyle{empty}
\makeatletter
\renewcommand{\section}{\@startsection{section}{1}{0mm}%
                                {-1ex plus -.5ex minus -.2ex}%
                                {0.5ex plus .2ex}%x
                                {\sffamily\large}}
\renewcommand{\subsection}{\@startsection{subsection}{2}{0mm}%
                                {-1explus -.5ex minus -.2ex}%
                                {0.5ex plus .2ex}%
                                {\sffamily\normalsize\itshape}}
\renewcommand{\subsubsection}{\@startsection{subsubsection}{3}{0mm}%
                                {-1ex plus -.5ex minus -.2ex}%
                                {1ex plus .2ex}%
                                {\normalfont\small\itshape}}
\makeatother
\setcounter{secnumdepth}{0}
\setlength{\parindent}{0pt}
\setlength{\parskip}{0pt plus 0.5ex}
% -----------------------------------------------------------------------

\usepackage{academicons}

\begin{document}

\definecolor{mathBlue}{cmyk}{1,.72,0,.38}
\definecolor{defOrange}{cmyk}{0, 0.5, 1, 0.3}
\definecolor{codeInlineRed}{cmyk}{0, 0.9, 0.9, 0.45}
\definecolor{theoremRed}{cmyk}{0, 0.9, 0.9, 0.45}
% Note: code and thm color are pretty similar, but typically they wont appear in the same file. Both are set to be dim so that the file is not overwhelmed by bright colors (since deff organe is quite bright).

\everymath{\color{mathBlue}}
\everydisplay{\color{mathBlue}}

% for vector notation in this module
\newcommand{\vect}[1]{\pmb{#1}}
\newcommand{\deff}[1]{\textcolor{defOrange}{\textbf{#1}}}
\newcommand{\codein}[1]{\textcolor{codeInlineRed}{\texttt{#1}}}
\newcommand{\citeqn}[1]{\underline{\textit{#1}}}
\newcommand{\thm}[1]{
    \color{theoremRed}{
        \uline{{\textbf{#1}}}
    }
    \color{black}
}

\footnotesize
%\raggedright

\setlength{\premulticols}{0pt}
\setlength{\postmulticols}{0pt}
\setlength{\multicolsep}{1pt}
\setlength{\columnsep}{1.8em}
\begin{multicols}{3}




% -----------------------------------------------------------------------
\begin{center}
\huge\bfseries MA2108S MT AY25/26
\end{center}
\section{1. Pre-requisites}
% \subsubsection{}
\thm{Well-Ordering Principle of $\mathbb{N}$} Every non-empty subset $S \in \mathbb{N}$ has a least (smallest) element.

\thm{Schröder-Bernstein Theorem} Let A and B be two sets. If $\exists$ injections $f: A\rightarrow B$ and $g: B\rightarrow A$, then there exists bijection $h: A\rightarrow B$



\section{2. The Real Numbers}
\subsection{Algebraic Properties, \textasciitilde}
Different types of means 
\begin{itemize}
    \item \deff{Arithmetic Means} $A_n = \frac{1}{n} \sum_{k=1}^n a_k$
    \item \deff{Geometric Means} $G_n = \Big( \prod_{k=1}^n a_k \Big) ^{1/n}$ 
    \item \deff{Harmonic Means} $H_n = n \Big( \sum_{k=1}^n a_k^{-1} \Big)^{-1}$
\end{itemize}
, for $n \in \mathbb{N}_{\ge 2}$ and $a_1, a_2, ..., a_n \in \mathbb{R}$ are positive. For the means, we have the \thm{AM-GM-HM Inequality}:
\[
H_n \le G_n \le A_n
\]
, taking "$=$" iff. $a_1 = ... = a_n$.



\thm{Bernoulli's Inequality} For $x > -1$, we have $(1+x)^n \ge 1 + nx$, for any $n \in \mathbb{N}$.



\thm{Triangle Inequity} $|a+b| \le |a| + |b|$, for all $a, b \in \mathbb{R}$.
\\
Derived: [1] $\bigl| |a| - |b| \bigr| \le |a-b|$, [2] $|a-b| \le |a|+|b|$.

\subsubsection{Neighbourhood}
For any $a \in \mathbb{R}$ and $\epsilon > 0$, the \deff{$\epsilon$-neighbourhood of $a$} is the set:
\[
V_\epsilon(a) = \{ 
x \in \mathbb{R}: |x - a| < \epsilon
\}
\]

\thm{Theorem 2.2.8} For $a \in \mathbb{R}$, if $x \in V_\epsilon(a)$ for every $\epsilon > 0$, then $x = a$.



\subsection{Completeness Properties, \textasciitilde}
For a non-empty $S \subseteq \mathbb{R}$, it is \deff{Bounded Above} (\deff{Bounded Below}) if $S$ has an upper bound (a lower bound). $S$ is \deff{Bounded} if it is bounded above and below, and is \deff{Unbounded}, otherwise.


\smallbreak


For a non-empty $S \subseteq \mathbb{R}$, $u$ is the \deff{Supremum} of $S$ if the following conditions are met, and we denote it as $\sup S$:
\begin{enumerate}
    \item $u$ is an upper bound of $S$.
    \item $\forall v \in \mathbb{R}$, if $v$ is an upper bound of $S$, then $v \ge u$.
\end{enumerate}
For a non-empty $S \subseteq \mathbb{R}$, $w$ is the \deff{Infimum} of $S$ if the following conditions are met, and we denote it as $\inf S$:
\begin{enumerate}
    \item $w$ is a lower bound of $S$.
    \item $\forall v \in \mathbb{R}$, if $v$ is a lower bound of $S$, then $v \le w$.
\end{enumerate}
\underline{Note}: Sup. and Inf. are \textbf{uniquely determined}, if they exist.


Alternative Definition (Similarly for Infimum):

\thm{Lemma 2.3.4} For $u$ an upper bound of $S \subseteq \mathbb{R}$, $u = \sup S$ iff.
\[
\forall \epsilon > 0, \exists s_\epsilon \in S, \ u-\epsilon < s_\epsilon 
\]



\smallbreak


For a non-empty $S \subseteq \mathbb{R}$, $u$ is the \deff{Maximum} (\deff{Minimum}) of $S$, if $u = \sup S$ ($u = \inf S$) and $u \in S$.
\\
\underline{Note}: Sup. and Inf. are not necessarily elements in $S$ (if they exist), but maximum and minimum are.



\thm{Supremum Property of $\mathbb{R}$} Every non-empty subset of $\mathbb{R}$ that has an upper bound has a supremum.



\thm{The Archimedean Property} If $x \in \mathbb{R}$, then $\exists n_x \in \mathbb{N}$ s.t. $x < n_x$.



\thm{Corollary 2.4.6} If $x > 0$, then $\exists n \in \mathbb{N}$ such that $n-1 \le x <  n$.

\thm{Corollary 2.4.7} $\forall x>0, \exists n \in \mathbb{N}$ s.t. $\frac{1}{n} < x$.

\thm{Density Theorems} For $x, y \in \mathbb{R}$ with $x < y$, there exists $r \in \mathbb{Q}$ ($z \in \mathbb{R} \backslash \mathbb{Q}$) s.t. $x < r < y$ ($x < z < y$).


% Supremum Property 

% Archimedean 

% Density Thm


\subsection{Intervals}
A sequence of intervals $I_n, n \in \mathbb{N}$ is \deff{Nested} if
\[
I_1 \supseteq I_2 \supseteq ... \supseteq I_n \supseteq I_{n+1} \supseteq ...
\]
\underline{Property}: If $I_n = [a_n, b_n], n \in \mathbb{N}$ satisfying $\inf \{ b_n - a_n: n \in \mathbb{N} \} = 0$, then $x$ contained in all $I_n$ is unique.

\thm{Nested Interval Property} Let $\{I_n=[a_n, b_n]:n\in\mathbb{N}\}$ be a sequence of closed intervals such that for any $n\in\mathbb{N}$, $I_{n+1} \subseteq I_n$. Then there exists $x\in\mathbb{R}$ s.t. $x\in\bigcap_{n=1}^\infty I_n$





\section{3. Sequences \& Series}
\subsection{Sequence \& Convergence}
% Seq def
\deff{Sequence} in $\mathbb{R}$: a real-valued function $X: \mathbb{R} \to \mathbb{R}$. We write $x_n = X(n)$ for the $n$-th term of the sequence, and denote the sequence as $(x_n,: n \in \mathbb{N})$.


\smallbreak


A sequence $X = (x_n)$ in $\mathbb{R}$ is \deff{Convergent} to $x \in \mathbb{R}$ iff. for every $\epsilon > 0$, there exists $K = K(\epsilon) \in \mathbb{N}$ s.t.
\[
n \ge K(\epsilon) \implies |x_n - x| < \epsilon
\]
, and we call $x$ the \deff{Limit} of $(x_n)$, denoted as $\lim_{n \to \infty} x_n = x$. A sequence is \deff{Divergent} if it is not convergent.


Technique for proving convergence:
\begin{enumerate}
    \item Express $|x_n - x|$ in terms of $n$ and find a simpler upper bound $L = L(n)$, i.e. $|x_n - x| < L$.
    \item Let $\epsilon > 0$ be arbitrary, find $K \in \mathbb{N}$ s.t. for all $n \ge K$, $L = L(n) < \epsilon$, then
    \[
        n \ge K \implies |x_n - x| < L < \epsilon
    \]
\end{enumerate}


\smallbreak


\thm{Squeeze Theorem} If $x_n \le y_n \le z_n$, for all $n \in \mathbb{N}$ and $\lim_{n \to \infty} x_n = \lim_{n \to \infty} z_n = a$, then
\[
\lim_{n \to \infty} y_n = a
\]



A sequence $X = (x_n)$ is \deff{Bounded} if there exists $M > 0$ such that $|x_n| \le M$ for all $n \in \mathbb{N}$.

% Thm 3.2.2 ~ 3.2.5 See fit


\smallbreak


\thm{Monotone Convergence Theorem} Let $(x_n)$ be a monotone sequence of real numbers, then $(x_n)$ is convergent iff. $(x_n)$ is bounded.
\\
If it is bounded and increasing, then $\lim_{n \to \infty} x_n = \sup \{ x_n: n \in \mathbb{N} \}$. (Similarly for decreasing.)


\smallbreak


For a sequence $(x_n)$, it \deff{tends to $+ \infty$}, i.e. $\lim_{n \to \infty} x_n = + \infty$ if for all $\alpha \in \mathbb{R}$, there exists $K = K(\alpha) \in \mathbb{N}$ such that if $n \ge K(\alpha)$, then $x_n > \alpha$. (Similarly for $\lim_{n \to \infty} x_n = - \infty$.)
\\
A sequence $(x_n)$ is \deff{Properly Divergent} if $\lim_{n \to \infty} x_n = \pm \infty$.




\subsection{Subsequences}
% Def
A \deff{Subsequence} of $X = (x_n)$ is $X' = (x_{n_k})$:
\[
X' = (x_{n_1}, x_{n_2}, ..., x_{n_3})
\]
, where $n_1 < n_2 < ... < n_k < ... $ is a strictly increasing sequence in $\mathbb{N}$. \underline{Note}: $n_k \ge n, \forall k$.


\thm{Theorem 3.4.2} If $(x_n)$ converges to $x$, then any subsequence $(x_{n_k})$ also converges to $x$, 
\[
\lim_{n_k \to \infty} x_{n_k} = \lim_{k \to \infty} x_{n_k} = x
\]


\thm{Theorem 3.4.5} If $(x_n)$ has either of the following properties, it is divergent: [1] $(x_n)$ has two convergent subsequences with different limits. [2] $(x_n)$ is unbounded.


\thm{Theorem 3.4.7} Every sequence has a monotone subsequence.


\thm{Bolzano-Weierstrass Theorem} Every bounded sequence has a convergent subsequence.




\subsection{Cauchy Sequences}
A \deff{Cauchy Sequence} $(x_n)$ is a sequence where for all $\epsilon > 0$, there exists $H = H(\epsilon) \in \mathbb{N}$ such that
\[
\forall n, m \in \mathbb{N}, n, m \ge H \implies |x_n - x_m| < \epsilon
\]

\thm{Cauchy Criterion} A sequence is convergent iff. it is Cauchy.


\smallbreak


A \deff{Contractive Sequence} $(x_n)$ is a sequence where there exists $C \in (0, 1)$ s.t. 
\[
|x_{n+2} - x_{n+1}| \le C |x_{n+1} - x_n|, \ \forall n \in \mathbb{N}
\]


\thm{Theorem 3.5.8} Every contractive sequence is Cauchy.

\subsection{Limit Points}
Suppose $\{a_n\}$ bounded from above. The \deff{upper limit} of $\{a_n\}$ is defined as
\[
\limsup_{n\rightarrow\infty} a_n := \lim_{n\rightarrow\infty} \sup\{a_k:k\geq n\}.
\]
Suppose $\{a_n\}$ bounded from below. The \deff{lower limit} of $\{a_n\}$ is defined as
\[
\liminf_{n\rightarrow\infty} a_n := \lim_{n\rightarrow\infty} \inf\{a_k:k\geq n\}.
\]
$A$ is called a \deff{limit point} of a sequence $\{a_n\}$ if there exists subsequence $\{b_k=a_{a_k}\}$ such that $\lim_{k\rightarrow\infty} b_k = A$. 
\thm{Theorem 3.6.1} Let $\{a_n\}$ be a bounded sequence and $E$ denote the set of limit points of $\{a_n\}$. Then both upper limit and lower limit of $\{a_n\}$ are contained in $E$. Moreover,
\[
\limsup_{n\rightarrow\infty} a_n = \max E, \liminf_{n\rightarrow\infty} a_n = \min E
\]

\thm{Corollary 3.6.2} For any bounded sequence $\{a_n\}$,
\[
\liminf_{n\rightarrow\infty} a_n \leq \limsup_{n\rightarrow\infty} a_n
\]
$\liminf_{n\rightarrow\infty} a_n = \limsup_{n\rightarrow\infty} a_n$ iff the sequence is convergent.

\subsection{Infinite Series}
For $(x_n)$, its \deff{(Infinite) Series} is sequence $(s_n)$, where $s_n = \sum{k=1}^n x_k$ is called a \deff{Partial Sum} of the series, and $x_k$ is a \deff{Term}.

Tests for infinite series' convergence:
\begin{itemize}
    \item \deff{$n$-th Term Test} - If $\sum x_n$ converges, then $\lim_{n\to\infty} x_n = 0$.
    
    \item \deff{Cauchy Criterion} $\sum_{i=1}^\infty a_i$ is convergent iff $\forall \epsilon > 0, \exists N\in\mathbb{N}$ such that for any $m > n \geq N$, $|a_{n+1}+...+a_m|<\epsilon$
    
    \item \deff{Partial Sum Bounded Test}, for series w. non-negative terms - Suppose $x_n \ge 0, \forall n \in \mathbb{N}$, then $\sum_{x_n}$ converges iff. $(s_n)$ is bounded.
    
    \item \deff{Comparison Test} - Suppose $\{a_n\}$ and $\{b_n\}$ satisfy that there exists $K\in\mathbb{N}$ and $\lambda>0$ such that for all $n\geq K$, $0\leq a_n \leq \lambda b_n$. Then [1] $\sum_{n=1}^\infty b_n$ convergent $\implies \sum_{n=1}^\infty a_n$ convergent, and [2] $\sum_{n=1}^\infty a_n$ divergent $\implies \sum_{n=1}^\infty b_n$ divergent.
    
    \item \deff{Limit Comparison Test} - For \textbf{strictly positive} $(a_n), (b_n)$ with limit $r = \lim_{n \to \infty} (\frac{a_n}{b_n})$. Then [1] if $r = 0$, $\sum b_n$ converges $\implies$ $\sum a_n$ converges, $\sum a_n$ diverges $\implies$ $\sum b_n$ diverges, [2] if $r > 0$, $\sum b_n$ converges iff $\sum a_n$ converges.

    \item \deff{Cauchy Condensation Test} - Suppose $\{a_n\}$ decreasing and every term is positive. Define $b_k:=2^ka_{2^k}$. $\sum_{n=1}^\infty a_n$ convergent iff $\sum_{n=1}^\infty b_n$ convergent.

    \item \deff{Leibniz Test} - Suppose $a_n \geq 0$ decreasing and $\lim_{n\rightarrow\infty} a_n = 0$. Then $\sum_{n=1}^{\infty}(-1)^{n - 1}a_n$ convergent. 

    \item \deff{Dirichlet's Test} Let $\{a_n\}$ satisfy that its partial sums $\{A_n=\sum_{k=1}^n a_n\}$ is bounded, and $\{b_n\}$ be a positive decreasing sequence approaching zero. Then $\sum_{n=1}^\infty a_nb_n$ is convergent.

    \item \deff{Abel's Test} Let $\{a_n\}$ satisfy that its partial sums $\sum_{n=1}^\infty a_n$ is convergent. Let $\{b_n\}$ be a monotone and bounded sequence. Then $\sum_{n=1}^\infty a_nb_n$ is convergent.
\end{itemize}





\subsection{Absolute Convergence}
% Def
Series $\sum x_n$ is \deff{Absolutely Convergent} if series $\sum |x_n|$ is convergent. A series is \deff{Conditionally Convergent} if it is convergent but not absolutely convergent.

Tests for absolutely convergence:
\begin{itemize}
    \item Limit Comparison Test - Consider convergence of positive sequences $|x_n|$ and $|y_n|$ if $(x_n), (y_n)$ non-negative.
    \item \deff{Root Test} - For $(x_n)$, [1] if $\exists r \in \mathbb{R}, 0 < r < 1$ and $K \in \mathbb{N}$ s.t. $|x_n|^{1/n} \le r, \ \forall n \ge K$, then $\sum x_n$ is abs. convergent. [2] If $\exists r \in \mathbb{R}, r > 1$ and $K \in \mathbb{N}$ s.t. $|x_n|^{1/n} \ge r > 1, \ \forall n \ge K$, then $\sum x_n$ is \textbf{divergent}.
    \item \deff{Ratio Test} - For $(x_n)$ nonzero, [1] if $\exists r \in \mathbb{R}, 0 < r < 1$ and $K \in \mathbb{N}$ s.t. $|\frac{x_{n+1}}{x_n}| \le r, \ \forall n \ge K$, then $\sum x_n$ is abs. convergent. [2] If $\exists K \in \mathbb{N}$ s.t. $|\frac{x_{n+1}}{x_n}| \ge 1, \ \forall n \ge K$, then $\sum x_n$ is \textbf{divergent}.

\end{itemize}

% Abs convg tests
% TODO: check if finishes

\subsection{Abel's Summation Formula}
Let $A_n = \sum_{k=1}^n a_n$.For any $q>p$, 
\[
\sum_{n=p}^q a_n b_n = A_qb_q - A_{p-1}b_p+\sum_{n=p}^{q-1}A_n(b_n-b_{n+1}).
\]


\subsubsection{Useful statements}
\[\sum_{i=1}^\infty \frac{1}{n^2} = \frac{\pi^2}{6}\]
\[\bigg(1+\frac{1}{n}\bigg)^n < e < \bigg(1+\frac{1}{n}\bigg)^{n + 1} \Rightarrow \frac{1}{n+1} < \log\bigg(\frac{n+1}{n}\bigg) < \frac{1}{n} \]

\[\sum_{n=1}^\infty \frac{1}{n\log n} \text{ is divergent}\]


\end{multicols}
\end{document}
